% This document is part of the Weak Lensing project
% Copyright 2013 the authors.

\documentclass[12pt,letterpaper]{article}

\begin{document}\sloppy\sloppypar

\section*{Weak lensing, beyond ellipticity}
\noindent{\raggedright
  {David W. Hogg} \\
  \textsl{Center for Cosmology and Particle Physics, Department of Physics, New~York~University, New~York, NY 10012 \\
          Max-Planck-Institut f\"ur Astronomie, Heidelberg, Germany}
}

\paragraph{abstract:}
The standard formulation of the weak-lensing estimation problem is that
  galaxies have their ellipticities measured,
  and then those ellipticity measurements are used to estimate something like
  a shear map, a shear--shear auto-correlation function, or a galaxy--shear cross-correlation function.
The key idea is that in the absence of lensing distortion,
  ellipticity orientations---angles of ellipticity eigenvectors---ought to be distributed isotropically;
  lensing is measured through deviations from this prediction.
Galaxies have complex morphologies,
  including bars, spiral arms, and HII regions.
All morphological features, especially those
  (like the angular displacements between pairs of intensity peaks)
  to which angular vectors can be assigned,
  ought also (in the absence of lensing) to be distributed isotropically.
Here we peform some expriments with toy and real images of galaxies,
  imaged with non-trivial, non-circular point-spread functions,
  that show that there is enormous potential untapped weak-lensing information
  latent in the complex morphologies of galaxies.
Extraction of this information requires, in general,
  full treatment of the point-spread function,
  because both the identification and the measurement of the relevant features
  can be exceedingly distorted if the point-spread function is modeled incorrectly.
However, we show that some of the new morphological measurements
  available for weak-lensing studies might be less dependent on the point-spread function model
  than standard galaxy ellipticity measurements.

\section{introduction}

\section{method}

\section{experiments}

\section{discussion}

\end{document}
