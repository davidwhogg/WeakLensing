\documentclass[12pt]{article}

\begin{document}

\paragraph{abstract:}
Imagine a toy weak-lensing Universe in which there is a finite sky
chopped up into $M$ equal-sized patches; in each patch $m$ there are
$N$ galaxies all distorted by a constant weak-lensing shear tensor
$\gamma_m$.  The cosmological model in this toy Universe is that each
of the $M$ shear tensors $\gamma_m$ is drawn independently from a
symmetric Gaussian centered on zero with a single scalar variance
$\Sigma^2$.  In this toy Universe, four cosmologists live and work;
they are all trying to determine the cosmological parameter
$\Sigma^2$, and they all agree on how to measure galaxy shapes, in the
sense that they all would write down the same likelihood function for
the galaxy ellipticity measurement given the same intrinsic galaxy
properties and same shear value.

Cosmologist A attempts to measure cosmological parameter $\Sigma^2$ as
follows: First make a point estimate for each galaxy ellipticity by
finding the maximum-likelihood ellipticity for that galaxy. Then
average---in some very sensible way---the $N$ galaxies in each patch
$m$ to get an estimate of the shear tensor $\gamma_m$ in patch $m$.
The estimate of the cosmological variance parameter $\Sigma^2$ is the
variance of the $M$ estimated shear tensors $\gamma_m$.

Cosmologist B infers the shear tensors $\gamma_m$ in a Bayesian way,
by asserting a prior probability density function on the intrinsic
unlensed galaxy shapes, and finding, in each patch, the shear tensor
$\gamma_m$ that best explains the imaging data.  Because the
individual galaxy measurements are nuisances, the procedure is to
marginalize out the $N$ individual galaxy measurements in each patch
using the intrinsic shape prior.  Cosmologist B takes the makes a
point estimate of each of the shear tensors $\gamma_m$ by taking the
maximum \emph{marginalized} likelihood for the shear in each patch.
The cosmological paameter $\Sigma^2$ is estimated by taking the
variance of the $M$ estimated shear tensors $\gamma_m$.

Cosmologist C does the same as cosmologist B but infers rather than
asserts the prior PDF on the intrinsic unlensed galaxy shapes.  This
involves hierarchical inference and assertion of a hyperprior, or
prior PDF on the possible prior PDFs for galaxy shapes.  For
cosmologist C, the marginalized likelihoods involve a marginalization
over both the individual galaxy shapes \emph{and} the prior on
intrinsic shapes itself.  Estimation of the cosmological parameter
$\Sigma^2$ proceeds as for cosmologists A and B.

Cosmologist D goes even further than cosmologist C by including the
cosmological parameter $\Sigma^2$ in the hierarchical model.
Everything is the same as for cosmologist C but now instead of
producing a point estimate for the shear $\gamma_m$ in every patch,
cosmologist D marginalizes out all the individual shear measurements
$\gamma_m$.  This is possible because the cosmological model sets the
prior PDF for the shears; it is the Gaussian whose variance is
parameter $\Sigma^2$.  This cosmologist produces a fully marginalized
likelihood for the cosmological variance parameter $\Sigma^2$ (which
is now a hyperparameter of the hierarchical model), marginalizing out
all the individual shape measurements, the intrinsic unlensed shape
prior PDF, and the individual-patch shear tensors.  This fully
marginalized likelihood can be optimized to obtain the maximum
marginalized likelihood value for the cosmological parameter
$\Sigma^2$, and evaluated over a region to get uncertainties; or it
can be combined with a hyperprior PDF (a prior on $\Sigma^2$) to
produce a fully marginalized posterior.

Questions include: Which cosmologist will do best in estimating the
cosmological parameter $\Sigma^2$ and why?  If some methods are
biased, why are they biased?  Do unbiased estimates of galaxy shapes
lead to unbiased estimates of the cosmological parameter?  If not, why
not?

\end{document}
