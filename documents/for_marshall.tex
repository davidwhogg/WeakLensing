\documentclass[12pt]{article}

\usepackage{color}

\def\shearvar{\langle |\bar{\gamma}|^2\rangle}
\def\mleps{\hat{\epsilon}}
\def\phil#1{\textcolor{blue}{#1}}

\begin{document}

\paragraph{Abstract:}
Imagine a toy weak-lensing Universe in which there is a finite sky chopped up
into $M$ equal-sized patches; \phil{in any given patch $m$ there are $N$
galaxies which are all being  distorted by a constant weak-lensing shear
tensor (in that patch), $\gamma_m$}.  The cosmological model in this toy
Universe is that each of the $M$ shear tensors $\gamma_m$ is drawn
independently from a symmetric Gaussian centered on zero with a single scalar
variance $\shearvar$.  In this toy Universe, four cosmologists live and work;
they are all trying to determine the cosmological parameter $\shearvar$, and
they all agree on how to measure galaxy shapes, in the sense that they all
would write down the same likelihood function for the galaxy ellipticity
measurement given the same intrinsic galaxy properties and same shear value.
\phil{In this toy Universe, it turns out that this likelihood is a simple
Gaussian function of each observed ellipticity component.}

\begin{itemize}

\item Cosmologist A attempts to measure cosmological parameter $\shearvar$ as
follows. \phil{They first make a point estimate for each galaxy ellipticity 
by finding the maximum-likelihood ellipticity $\mleps$ for that
galaxy.} 
They then average the $N$ galaxies in each
patch $m$ to get an estimate $\bar{\gamma}_m$ 
of the shear tensor in patch $m$. \phil{However, 
as e.g. Schneider (2006) points out, the square of this quantity is not
an unbiased estimator of $\shearvar$. Instead, 
they estimate the cosmological shear variance
using the following formula (see e.g Schneider 2006):
\begin{equation}
  \shearvar_{A} = \frac{1}{M(M-1)}\sum^{M}_{i \neq j} \gamma_i \gamma^{*}_j
\end{equation}
% The estimate of the cosmological variance parameter $\shearvar$ is the
% variance of the $M$ estimated shear tensors $\gamma_m$.
As Schneider notes, this estimator is not positive semi-definite, but it is
expected to be unbiased.}\footnote{\phil{Actually, the toy model that corresponds
best to Schneider's discussion of shear variance would be to 
draw every galaxy's shear from a
Gaussian distribution. Then, Cosmologist A would
attempt to measure cosmological parameter $\shearvar$ as
follows. They would first make a point estimate for each galaxy ellipticity 
by finding the maximum-likelihood ellipticity $\hat{\epsilon}$ for that
galaxy. 
Then they would estimate the shear variance in each patch from the
ellipticities directly
using the following formula:
\begin{equation}
  \shearvar_{A,m} = \frac{1}{N(N-1)}\sum^{N}_{i \neq j} \mleps_i \mleps^{*}_j
\end{equation}
And then they would simply average the shear variance estimates over the
ensemble of patches.}}


\item Cosmologist B infers the shear tensors $\gamma_m$ in a Bayesian way,
by asserting a prior probability density function on the intrinsic
unlensed galaxy shapes, and finding, in each patch, the shear tensor
$\gamma_m$ that best explains the imaging data.  Because the
individual galaxy measurements are nuisances, the procedure is to
marginalize out the $N$ individual galaxy measurements in each patch
using the intrinsic shape prior.  Cosmologist B makes a
point estimate of each of the shear tensors $\gamma_m$ by taking the
maximum \emph{marginalized} likelihood for the shear in each patch.
The cosmological paameter $\shearvar$ is estimated by taking the
variance of the $M$ estimated shear tensors $\gamma_m$.

\item Cosmologist C does the same as cosmologist B, but infers, rather than
asserts, the prior PDF on the intrinsic unlensed galaxy shapes.  This
involves hierarchical inference and assertion of a hyperprior, or
prior PDF on the possible prior PDFs for galaxy shapes.  For
cosmologist C, the marginalized likelihoods involve a marginalization
over both the individual galaxy shapes \emph{and} the prior on
intrinsic shapes itself.  Estimation of the cosmological parameter
$\shearvar$ proceeds as for cosmologists A and B.

\item Cosmologist D goes even further than cosmologist C by including the
cosmological parameter $\shearvar$ in the hierarchical model \phil{as well.}
Everything is the same as for cosmologist C, except now, instead of
producing a point estimate for the shear $\gamma_m$ in every patch,
cosmologist D marginalizes out all the individual shear measurements
$\gamma_m$.  This is possible because the cosmological model sets the
prior PDF for the shears; it is the Gaussian whose variance is
parameter $\shearvar$.  This cosmologist produces a fully marginalized
likelihood for the cosmological variance parameter $\shearvar$ (which
is now a hyperparameter of the hierarchical model), marginalizing out
all the individual shape measurements, the intrinsic unlensed shape
prior PDF, and the individual-patch shear tensors.  This fully
marginalized likelihood can be optimized to obtain the maximum
marginalized likelihood value for the cosmological parameter
$\shearvar$, and evaluated over a region to get uncertainties; or it
can be combined with a hyperprior PDF for the parameter $\shearvar$ to
produce a fully marginalized posterior PDF.

\end{itemize}

In this investigation we ask:
\begin{itemize}
\item Which cosmologist will produce the \phil{
most accurate estimate (or inference)}
of the cosmological parameter $\shearvar$, and why? 
\item If some methods are biased \phil{(that is, systematically lead to the
recovery of the wrong value of $\shearvar$, compared to the one that was input
when generating the toy dataset)}, why are they biased?  
\item Do unbiased estimates of galaxy shapes lead to unbiased estimates of the
cosmological parameter?  If not, why not?
\item \phil{If some methods are less precise than others, why is that the 
case?}
\end{itemize}

\end{document}
