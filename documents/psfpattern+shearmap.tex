% This document is part of the Weak Lensing project
% Copyright 2013 the authors.

\documentclass[12pt,letterpaper]{article}

\begin{document}\sloppy\sloppypar

\section*{Infer the PSF pattern simultaneously with the Shear Map}
\noindent{\raggedright
  {Phil Marshall} \\
  \textsl{Kavli Institute for Particle Astrophysics and Cosmology, Stanford University}
}

\paragraph{abstract:}

The traditional approach to weak lensing is to measure galaxy shapes in the
presence of anisotropic non-circular point spread function (PSF) distortions
by first estimating the PSF distortions from the bright stars in the image,
and then correcting the galaxy shapes before averaging them together in a
shear estimate. Here, we investigate the possibility of simultaneously
inferring both the shear map and the PSF distortion map from all measured
objects. We expect to gain information on the PSF distortions from the large
numbers of small, faint objects (both stars and galaxies) in the field, that
would normally be rejected in order to try and keep the star and galaxy
catalogs clean. In general, this inference involves two flexible
two-dimensional functions for capturing the spatial variation of both the PSF
and shear patterns, and a flexible function for capturing the intrinsic
distribution of galaxy size and ellipticity, all of which are simultaneously
inferred. Each object has a probability of being a galaxy, that must be
inferred, and also comes with measured magnitude and colors that can be used
in the inference. We show how the small galaxies and faint stars improve the
shear map accuracy and precision, and quantify the improvements brought by
using the colors to influence the star-galaxy separation. We do not consider
color or magnitude-dependent PSFs, but discuss how these could be included in
the inference.


\section{introduction}

\section{method}

\section{experiments}

\section{discussion}





\end{document}
