\documentclass[12pt]{article}

\begin{document}

\noindent\textsl{This is a draft document by David W. Hogg (NYU \& MPIA) dated 2012-06-14.}

\paragraph{background:}
On 2012-05-31, Konrad Kuijken mentioned to me that he is working on
novel approaches to weak lensing and that he is committing the sin of
smoothing images to get a uniform, circular psf.  He noted that his
``smoothing'' is done with a kernel that can contain negative values
so it sometimes is used to lightly deconvolve (increase angular
resolution), sometimes to lightly convolve (decrease angular
resolution), but always to leave a circular mean psf.  I conjectured
that there is a \emph{fitting procedure} that would be better than
smoothing; that is, better than applying a linear filter without
regard to a noise model or a likelihood function.  The point of this
document is to make this conjecture more well posed.

The way Kuijken uses his smoothing is (roughly) as follows (if I
understood him correctly): He takes a weak-lensing shear map, and
applies \emph{its inverse} to the image.  He determines the psf (or a
smoothing kernel that will circularize the stars) in the
inverse-sheared image.  He checks that the mean galaxy ellipticity in
this inverse-sheared and smoothed image is zero.  If it isn't, he
updates the shear map (presumably cleverly, using gradients) and tries
again.  Importantly for what follows (if I understood correctly),
Kuiken does \emph{not} need to know the psf in the image; he only
needs to know the kernel that circularizes it.  That's key.

Of course it goes without saying that the kernel will vary in the
field---hell, he is inverse-shearing before estimating it---but how
the kernel is permitted to vary is a model-complexity detail that
doesn't affect any of the below.  At least I think that's true.

\paragraph{setup:}
Imagine that you have an image $D_n$ (in Kuiken's case an
inverse-sheared image) of a patch of sky, and \emph{you don't yet
  know} the psf $\psi_n$.  I am going to assume that all your
information about the psf $\psi_n$ for image $n$ comes from image $n$
itself; that is, I am going to assume that you don't have any
additional external data that tell you the psf.  That must be true,
right?

I am going to make the stronger assumption that within image $D_n$
there are a set sources that you are pretty confident are either stars
or unresolved galaxies.  That is, I am going to assume that you know
that at $J$ two-dimensional positions $x_{nj}$ there are $J$ point
sources $j$.  These stars are what you would normally use to infer the
psf.  Indeed, although this assumption (known stars) is strong, unless
you are doing something very clever, I think your psf comes from some
kind of assumption about some sources being stars anyway, which is
equivalent to this.

Now, interestingly, for the Kuijken methodology, Kuijken does
\emph{not} need to know the psf $\psi_n$.  He just needs to be able to
make it circular.  The way anyone normal would think about this is
that he wants to find a kernel that can be applied to (convolved with)
the image $D_n$ to make a new image $D_\ast$ that has a standard,
circular, known psf $\psi_\ast$.  Ordinarily he could go about this by
using the known stars at the positions $x_{nj}$ to infer the psf
$\psi_n$ and then use the relationship between $\psi_n$ and
$\psi_\ast$ to get the kernel and smooth away.

Because I think of this as an inference problem, I prefer to think of
the \emph{inverse kernel}.  I want to find the kernel $K_n$ that could
be used to take the circular-psf image $D_\ast$ to the real-psf image
$D_n$.  I then want to \emph{infer} the circular-psf image using
probabilistic inference (or an approximation to it).

\paragraph{model and conjecture:}
Getting more specific, my model is
\begin{eqnarray}\displaystyle
D_n = K_n \ast D_\ast + e_n \quad,
\end{eqnarray}
where $D_n$ and $D_\ast$ are two-dimensional pixellated images, $K_n$
is a two-di\-men\-sion\-al kernel (possibly with some negative elements),
$\ast$ is the convolve operator, and $e_n$ is a two-dimensional image
of noise.  If the noise is seen as being drawn from some distribution
(like a Gaussian with zero mean and some variance $C_n$), then this
model implies a likelihood function and a fitting
objective---badness-of-fit scalar---something like
\begin{eqnarray}\displaystyle
\chi^2 = \sum_{\mathrm{pixels}} C_n^{-1}\,[D_n - K_n \ast D_\ast]^2 \quad,
\end{eqnarray}
where implicitly $C_n^{-1}$ is also an image like $D_n$, but an image of inverse variances or fitting weights.  I am being loose
with the math here.

This model has more parameters than data, of course, because both
$K_n$ and $D_\ast$ are unknown images, and the latter is as large as
$D_n$.  Not to worry, we will regularize with a very informative
prior.  The prior is that the circular-psf image $D_\ast$, \emph{in
  the vicinity of the stars at $x_{nj}$}, look like delta-functions
convolved with the standard psf $\psi_\ast$.  That is, we can add to
the badness-of-fit scalar a penalty like
\begin{eqnarray}\displaystyle
\chi_r^2 = \chi^2 + \sum_{\mathrm{pixels}} w_n\,[D_\ast - \psi_\ast \ast \sum_j S_j\,\delta(x - x_{nj})]^2 \quad,
\end{eqnarray}
where $\chi_r^2$ is the regularized objective, $w_n$ is a weight map
the same size as the image $D_\ast$ with weights set to zero far from
star positions $x_{nj}$ and set to some large value near those
positions, $\psi_\ast$ is the standard, uniform, circular psf, the
$S_j$ are the brightnesses of the stars $j$, and $\delta(\cdot)$ is
the two-dimensional delta-function.

The model has several free parameters: The investigator can choose how
to generate the inverse variance map $C_n^{-1}$.  There is a radius
around each star position $x_{nj}$ out to which the weight map $w_n$
is non-zero.  There is a non-zero weight value for the weight map near
the stars.

My suggestion is to find the image $D_\ast$ and kernel $K_n$ (and star
brightnesses $S_j$) that jointly optimize the regularized objective
$\chi_r^2$.  My conjecture is that this will perform better than any
straight smoothing.

\paragraph{notes:}
Because I recommend moving from the smoothing kernel that goes from
$\psi_n$ to $\psi_\ast$ to a kernel $K_n$ that goes from $\psi_\ast$
to $\psi_n$, I implicitly recommend making the standard circular psf
$\psi_n$ slightly (very slightly) \emph{narrower} than $\psi_n$.  To
keep things well behaved you usually want to be inferring generally
(though not exclusively) non-negative kernels, and therefore you want
to lightly \emph{deconvolve} when you infer the image $D_\ast$.  That
might seem counter-intuitive, but hey.

\end{document}
