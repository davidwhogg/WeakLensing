\documentclass[12pt]{article}
\begin{document}

\section*{Hierarchical inference of cosmic shear maps}
\noindent
David W. Hogg \\
\textsl{Center for Cosmology and Particle Physics, New York University} \\
and \textsl{Max-Planck-Institut f\"ur Astronomie} \\[1ex]
Phil Marshall \\
\textsl{Oxford University}

\begin{abstract}
We set up a toy weak-lensing problem---an overly simple scenario in
which both the two-dimensional morphologies of galaxies under study
and the imaging point-spread function are Gaussian---and look at the
methods by which a shear map can be inferred.  While it is true that
the average ellipticity of a sufficiently large sample of observed
galaxy images does provide an estimate of the cosmological shear, we
show that it is much more precise and accurate to probabilistically
infer the shear map from the distribution of observations.  This
inference is hierarchical in that it involves learning the parameters
of an exceedingly flexible model for the distribution of unlensed
morphologies, subject to a constraint that this unlensed distribution
has no mean ellipticity.  Although the problem we solve here is a toy,
all real issues with real data are expected to amplify the relative
value of hierarchical probabilistic inference over brute sample
averaging.  We make no representations, however, about computational
costs.
\end{abstract}

...content here...

\end{document}
