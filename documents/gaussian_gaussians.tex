% This file is part of the Weak Lensing project.
% Copyright 2011, 2012 David W. Hogg (NYU) and Phillip J. Marshall (Cambridge).

\documentclass[12pt]{article}
\begin{document}

\section*{Hierarchical inference of cosmic shear maps}
\noindent
David~W.~Hogg \\
\textsl{Center for Cosmology and Particle Physics, New York University} \\
and \textsl{Max-Planck-Institut f\"ur Astronomie} \\[1ex]
Phillip~J.~Marshall \\
\textsl{Oxford University}

\begin{abstract}
We set up a toy weak-lensing problem---an overly simple scenario in
which both the two-dimensional morphologies of galaxies under study
and the imaging point-spread function are Gaussian---and look at the
methods by which a shear map can be inferred.  While it is true that
the average ellipticity of a sufficiently large sample of observed
galaxy images does provide an estimate of the cosmological shear, we
show that it is much more precise and accurate to probabilistically
infer the shear map from the distribution of observations.  This
inference is hierarchical in that it involves learning the parameters
of an exceedingly flexible model for the distribution of unlensed
morphologies, subject to a constraint that this unlensed distribution
has no mean ellipticity.  Although the problem we solve here is a toy,
all real issues with real data are expected to amplify the relative
value of hierarchical probabilistic inference over brute sample
averaging.  We make no representations, however, about computational
costs.
\end{abstract}

...content here...

Introduction argument: Imagine (unrealistically) that all galaxies
have the same non-trivial non-circular shape, and are all exemplars of
that same shape, but with different orientations on the sky.  Then
even the observation of a \emph{single} galaxy would provide a
significant local measure of the shear map.  Taking ellipticities and
averaging would be wasteful and imprecise relative to two-dimensional
morphological modeling.  For example, what if all unlensed galaxies in
some color-selected sample are precisely two-dimensional Gaussians
that are 2 arcsec along the major axis and 1 arcsec along the minor
axis.  That's too simple for reality---Galaxies live in
three-dimensional space, after all, so there are two other Euler
angles---but it shows that averaging can never capitalize on
informative features in the distribution of morphologies.

Another introduction argument: Show (if it is indeed possible to do
so) that if the distribution of morphologies is ``featureless'' in the
relevant ways, hierarchical inferenct reduces to averaging.  Is that
true or possible?

Define terms, especially ``What constitutes a \emph{shear map}?''

\section{toy model}

\section{hierarchical method}

\section{results}

\section{discussion}

\end{document}
