% This document is part of the Weak Lensing project
% Copyright 2013 the authors.

\documentclass[12pt,letterpaper]{article}

\begin{document}\sloppy\sloppypar

\section*{Weak lensers: Infer the point-spread function from the galaxies}
\noindent{\raggedright
  {David W. Hogg} \\
  \textsl{Center for Cosmology and Particle Physics, Department of Physics, New~York~University, New~York, NY 10012 \\
          Max-Planck-Institut f\"ur Astronomie, Heidelberg, Germany}
}

\paragraph{abstract:}
The standard operating procedure for weak lensing studies is
  to infer the point-spread function (PSF) from the stars in each image,
  and interpolate that PSF information to the galaxy positions,
  where the weak-lensing signal is measured.
There are many issues with this procedure, including that
  the stars have different colors than the galaxies (and the PSF is wavelength-dependent),
  the (useful) stars are sparse in the images,
  and interpolation involves assumptions and choices.
Here we recommend using the \emph{galaxies themselves} to perform the PSF inference.
The key idea is that in most imaging surveys in which PSF estimation is required,
  the same galaxies are observed through differing PSFs in the imaging taken
  under different conditions.
The PSF-deconvolved galaxy shapes and the imaging PSFs can, in principle,
  be inferred simultaneously.
We perform experiments with toy data demonstrating the feasibility of these ideas,
  and their promise for current and future weak-lensing surveys.
One constraint for projects of this type is that if the PSF is to be
  inferred entirely from the galaxies,
  and if the instrumental component of the PSF is not perfectly understood,
  observing strategies (dither patterns) optimal for self-calibration must be adopted.

\section{introduction}

\section{method}

\section{experiments}

\section{discussion}

\end{document}
